%% Задание
%%% Техническое задание и исходные данные к работе
\technicalspec{По известному множеству наборов данных необходимо научиться классифицировать поступающие новые наборы данных, чтобы понимать к какой из уже известных категорий они относятся.   }

%%% Содержание выпускной квалификационной работы (перечень подлежащих разработке вопросов)
\plannedcontents{Пояснительная записка должна демонстрировать наиболее правильный подход к решению этой задачи, а также его плюсы и минусы по сравнению с другими методами. Должно быть произведено сравнение оптимальности ответов, времени работы, а также других метрик для всех рассмотренных способов классификации.}

%%% Исходные материалы и пособия 
\plannedsources{Не использовались.}

%%% Календарный план
\addstage{Ознакомление с предметной областью}{12.2016}
\addstage{Проработка идеи решения}{01.2016}
\addstage{Работа с исходными наборами данных}{02.2017}
\addstage{Реализация основных признаков}{03.2017}
\addstage{Проведение расчетов и сравнение результатов различных классификаторов}{04.2017}
\addstage{Написание пояснительной записки}{05.2017}

%%% Цель исследования
\researchaim{Научиться более оптимально решать описанную задачу.}

%%% Задачи, решаемые в ВКР
\researchtargets{\begin{enumerate}
    \item Провести исследование описанной задачи;
    \item Привести исходные наборы данных к более читаемому программному виду;
    \item Подобрать максимальное количество признаков (features).
    \item Внедрить различные классификаторы и посмотреть, как можно улучшить базовое решение.
\end{enumerate}}

%%% Использование современных пакетов компьютерных программ и технологий
\advancedtechnologyusage{Для реализации решения задачи был использован язык программирования Java 1.8 с дополнительно установленной библиотекой Weka для обучения и тестирования различных классификаторов.
 Также использовался Mission Control Center для перевода файлов из формата excel в json. }

%%% Краткая характеристика полученных результатов 
\researchsummary{Результаты, полученные в данной работе могут быть использованы в программном обеспечении компанией VeeRoute для классификации новых поступающих наборов данных.}

%%% Гранты, полученные при выполнении работы 
\researchfunding{При выполнении работы грантов получено не было.}

%%% Наличие публикаций и выступлений на конференциях по теме выпускной работы
\researchpublications{По теме данной работы публикаций и выступлений на конференциях нет.}

%% Макрос для введения. Совместим со старым стилевиком.
\startprefacepage

\section{Краткое описание}

В последнее время очень важной становится задача обработки новых данных.
С течением времени у каждой компании  появляются определенные наборы данных.
Какие-то получаются на основании полученных результатов, какие-то пишут тестировщики вручную.
Каждые такие наборы чаще всего можно объединить с какими-то другими, получается некое множество различных категорий.
Потом компания получает какой-то новый набор данных, и она хочет узнать, к какой категории из имеющихся можно отнести эти новые данные, поскольку, скорее всего, компания уже имеет какой-то план действий, как поступать с информацией того или иного типа.

Для решения этой задачи будет использоваться машинное обучение.
Из имеющихся данных, предоставленных компанией VeeRoute, будут сформированы различные признаки (features), по которым можно будет грамотно определить принадлежность к тому или иному классу.
Далее следует выделить обучающуюся и тестовую выборки.
На основании известных данных будет обучаться классификатор.
Типов классификаторов в машинном обучении имеется достаточно множество, в приведенной работе будут использоваться некоторые с помощью библиотеки Weka для Java.
Получившаяся модель будет тестироваться на новых данных, чтобы понять, насколько выбранный тип классификации оптимально определяет класс объектов. 




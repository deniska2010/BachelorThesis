\section{Актуальность}

Актуальность исходной задачи проявляется в необходимости многих компаний работать с новыми данными, на основании того, к какому классу из имеющихся они относятся (или не относятся ни к какому).

Разумеется эта работа будет очень полезна для компании VeeRoute, поскольку основная задача этой компании ~--- оптимизация логистических процессов для внутригородского транспорта, будь то перевозка грузов или просто такси.
Компании приходится сталкиваться с постоянным притоком новых наборов данных (datasets).
Таким образом, очень важно быстро и качественно уметь классифицировать новые datasets, чтобы ускорить работу всех отраслей компании в целом.

Каждый набор данных ~--- какая-то бизнес-задача, с которой обращаются в компанию заказчики. VeeRoute уже имеет представление, что делать с известными ей данными и какие к ним алгоритмы применять для построения или оптимизации маршрута. Поэтому, если уметь грамотно определять принадлежность новых данных к какому-то старому известному классу, то можно применять известный алгоритм. В противном случае, придется проводить дополнительные исследования, что конечно займет неопределенное время. В этом и заключается актуальность этой работы для компании VeeRoute

%\section{Цель}

%Цель этой бакалаврской работы~--- изучить поставленную задачу и ее текущие методы решения, предложить более эффективный метод решения задачи.

\section{Новизна}

Ранее написанных классификаций информации в компании не существовало.
Поскольку с момента основания компании (2013 ~--- 2014) данных стало намного больше, то сильно возросла необходимость в грамотном определении того, к чему относится тот или иной набор данных.

С точки зрения научной новизны, тяжело сказать что-то новое, поскольку никаких новых способ или алгоритмов не изобретается.
Работа представляет из себя инженерную разработку.
Единственное,что можно сказать ~--- поднимается вопрос того, как корректнее определять принадлежность к тому или иному классу в условиях не очень большой тестовой выборки  ~--- основная задача машинного обучения.


\section{Структура работы}

В главе $1$ представлен обзор предметной области.
Говорится о подробностях нашей задачи, что она из себя представляет с инженерной точки зрения.
Также подробным образом объясняется, какую именно бизнес-задачу решает данное исследование.
И насколько это будет полезно для компании VeeRoute.
Подробно изучена структура исходных наборов данных.
Расписано, как они были преобразованы из excel в json-формат.
Упоминается про группы данных, которые мы будем использовать в машинном обучении в рассматриваемой задачи.

В главе $2$ представлено примерное решение исходной задачи.
Указаны features, которые были использованы для обучения классификатора и их (признаков) подробное описание.
Как внедряется машинное обучение, какие типы классификаторов используются и их описание. Сводится ли многоклассовая классификация к бинарной и тому подобные вопросы возникающие при решении данной задачи.

В главе $3$ представлены программные результаты работы на различных данных и на разных моделях.
Показаны, какие признаки влияли на классификацию и какие выводы вообще можно сделать в итогам исследования.


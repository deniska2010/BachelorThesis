%% Начало содержательной части.
\chapter{Обзор предметной области }

В этой главе рассказывается про суть задачи, вводятся нужные определения, описываются входные данные и их типы, а также поднимается вопрос того, какую бизнес-задачу решает компания и зачем ей нужна эта работа.

\section{Краткое описание задачи}

Как уже было сказано во введении, определение типа новых наборов данных ~--- важнейшая задача в инженерной разработке.
Грамотное определение принадлежности к определенной категории многократно увеличивает скорость обработки данных в компании.
В данном случае эта задача решается для компании, специализирующийся на оптимизации логистических процессов для внутригородского транспорта.

Решение этой задачи заключается в машинном обучении, а точнее в создании грамотного программного алгоритма, который и будет классифицировать наборы данных.

Для решения задачи будут использоваться данные, предоставленные компанией VeeRoute.
Опираясь на них, будут сформированы признаки, по которым далее будет происходить машинное обучение.
По признакам будет построен классификатор, который и будет решать поставленную задачу.
Основная первоначальная цель ~--- достигнуть хорошего результата на предоставленных данных.

\section{Какую обширную задачу решает себе компания данным исследованием?}

Теперь несколько слов о важности этой работы для компании VeeRoute.
Выше уже говорилось, что это такое место, которое производит enterprise software по управлению логистики.
В основном, это, конечно же, доставки различных вещей и грузов.

Очень важно понимать разницу между бухгалтерией и логистикой.
Это совсем разные понятия.
В бухгалтерии существуют определенные законы и регламенты, все делается, руководствуясь этими вещами.
В логистики же нет ничего такого, нет никаких четких правил.

В число бизнес-процессов логистики входят:
\begin{enumerate}
\item планирование движения продукта;
\item доставка продукта от производителя или поставщика;
\item ведение складского учета полученного груза ;
\item доставка товара в магазины;
\item контроль за движением товарных потоков.
\end{enumerate}
Следует также понимать,что бизнес-процессы логистики в любой крупной бизнеса всегда уникальны.
В каждой компании все происходит абсолютно по-разному.
Так как в любой организации есть уникальные особенности бизнес-процесса , то нет какого-то единого универсального бизнес-решения.
Это значит, что наборы данных, по факту, у каждой компании могут быть полностью не похожими на другие.
То есть бизнес-логика, решение и требование к нему, данные у каждой крупной организации абсолютны уникальны.
Это приводит к тому, что и входные наборы данных (и их формат) у каждой компании свои собственные, исключительные.
Сказанное выше приводит к тому, что перед тем, как запустить поиск какого-то решения на клиентском датасете, необходимо валидировать эти данные на их корректность, поскольку нету единого стандарта.

В чем состоит исходная задача.
Уже имеются известные наборы решений на датасеты, которые тоже уже известны.
Появляется новый клиент со своими собственными данными.
Компании нужно понять:
\begin{enumerate}
	\item валидные ли данные или нет;
	\item есть ли уже похожие данные, под которые есть готовые решения.
\end{enumerate}
Под вторым пунктом подразумевается готовый набор конфигурационных файлов, учитывающих специфику бизнес-процессов, и особенность входных данных.
Суть в том, что, имея такой алгоритм, при добавлении изначального датасета каждого нового клиента  и configuration файла, можно сделать самообучающуюся систему, которая в зависимости от того, какой пришел новый клиент: 
\begin{enumerate}
	\item проверит валидность данных;
	\item подберет похожий набор конфигурационных файлов, который уже решает такую задачу.
\end{enumerate}
На выходе получается очень важная бизнес-задача.
В данный момент она все-таки чуть менее привлекают компанию VeeRoute.
Потому что это достаточно трудоемкая бизнес-задача и данных пока не так много.
Пока еще это все можно сделать вручную, но по мере роста компании вопрос станет все более и более актуальным.
Чем больше будет приток клиентов, тем тяжелее компании будет с этим справляться без специального софта.

Конечно, можно сказать, что всегда в логистических организациях  решается задача коммивояжера или ее производная, но можно привести вот такие примеры (специфики):
\begin{enumerate}
	\item у одной компании доставка по большому городу (например, Москва) с наличием только одного склада,а у другой шесть таких мест;
    \item у одной компании только клиентская доставка, а в другой есть еще и забор (Pickup and Delivery). В третьей компании, при наличии определенного товара, нужно автоматически добавлять новую точку в маршруте.
\end{enumerate}
В обоих случаях это абсолютно разные бизнес-задачи и решаются они диаметрально противоположно.
Каждая бизнес-логика принципиально влияет на набор датасетов и конфигурационный файл на платформе.


\section{Классы бизнес-задач}

Согласно API (данные) планировщик может получить для решения очень сложную комбинаторную задачу. Поскольку сложные задачи требуют сложных подходов, сложную модель вычислений, трепетное отношение к качеству результата и очень сложное пространство решений, во многих случаях расчет без дополнительных предположений будет требовать намного больше времени и вдобавок качество решения будет хуже, чем если бы эти предположения были введены. Поскольку кроме формальной модели, представленной в API, есть еще и человеческие ожидания, а вследствие этого и неявные предположения, в планировщике используется понятия класса задачи, решающее проблему формализации человеческих ожиданий.

Класс задачи есть система дополнительных ограничений, которые чисто формально следуют из постановки задачи (например, что исполнитель не может выполнить более 100 заказов за рабочую смену исходя из продолжительности заказов). Планировщик при старте расчета определяет класс задачи и в дальнейшем использует его в процессе расчета для различных оптимизаций (качество, скорость, разные наборы решений для разных классов задач, сам процесс расчета может отличаться для разных классов задач). В то же время это усложняет тестирование планировщика, поскольку на разных классах задач его поведение может существенно отличаться.

Система дополнительных ограничений состоит из конкретных, достаточно простых свойств, которые должны быть понятны человеку и в большинстве случаев легко диагностируемы <<на глаз>>.

При старте расчета планировщик выводит в лог подробную информацию обо всех свойствах, а также определенный на основании свойств тип задачи.

Имеются следующие классы задач:

\begin{itemize}
\item Delivery (развозка с одного склада)
\item Multiple Depots (развозка с нескольких складов)
\item Service Engineers (сервисные инженеры / обслуживание объектов)
\item Pickup and Delivery
\end{itemize}

\subsection{Pickup and Delivery}

Для задачи типа <<Pickup and Delivery>> нет каких-либо специфичных свойств. Это самый общий класс задач, любая задача по умолчанию относится к этому классу.

\subsection{Service Engineers}

Этот тип задачи требует следующего набора свойств:

\begin{enumerate}
\item Простые приоритеты
\begin{enumerate}
\item haveOnlySimplePrecedences	

Отсутствие сложных приоритетов (haveOnlySimplePrecedences) ~--- у работы (Work) приоритет внутри рейса равен 0, у забора (Pickup) равен 1, у доставки (Drop) равен 2. Приоритеты внутри заказа для всех заявок равны 0. Соответствует тому, что в рамках любого маршрута все заборы будут раньше всех доставок (на работы ограничений нет).
\end{enumerate}
\item Каждый заказ есть ровно одна заявка типа Work
\begin{enumerate}
	\item haveOnlyWorkDemands
	
	Все заявки только типа Work
	\item haveNoConsolidatedDemands
	
	Ровно одна заявка в любом заказе 
	\item haveNoMultiEvents	
	
	Наличие ровно одного события в рамках одной любой заявки.
\end{enumerate}
\item Все заявки достаточно продолжительны по времени
\begin{enumerate}
	\item allEventsAreLongLasting
	
	Продолжительность любого события составляет не менее 10\% времени любой смены, в которой он может быть выполнен.
	\end{enumerate}
\end{enumerate}

\subsection{Multiple Depots}

Этот тип задачи требует следующего набора свойств:

\begin{enumerate}
	\item Простые приоритеты
	\begin{enumerate}
		\item haveOnlySimplePrecedences	
		\item mustAllPickupsBeBeforeDrops
		
		Любой забор раньше любой доставки. Следует из свойства haveOnlySimplePrecedences, но может выполняться и в других случаях.
	\end{enumerate}
	\item Все заказы имеют только одну простую заявку на забор и развозку одного груза
	\begin{enumerate}
		\item haveOnlyWorkDemands
		\item haveNoConsolidatedDemands
		\item haveNoMultiEvents	
		\item doAllPickupsHaveExactlyOneCargo
		
		Во всех заборах фигурирует ровно один груз.
	\end{enumerate}
	\item Нет сложных временных ограничений
	\begin{enumerate}
		\item doAllPickupTimeWindowsCoincide
		
		У всех событий всех заборов мягкие (Soft) и жесткие (Hard) временные окна совпадают между собой и одни и те же у всех заборов (например, с 6 до 20).
		\item haveNoComplicatedLocationTimeWindows
		
		Для каждой заявки есть временное окно работы локации, в рамках которого заявка целиком выполнима.
		\item doAllTransportsHaveSameShifts
		
        У всех транспортов одинаковые окна смен. Если взять два разных транспорта, то смен у них поровну и в соответствующих сменах временные окна смен у них совпадают.
		\item doAllPerformersHaveSameWorkShifts
		
		У всех исполнителей одинаковые рабочие смены. Если взять два разных исполнителя, то смен у них поровну и в соответствующих сменах временные окна смен у них совпадают.	
	\end{enumerate}
	\item А также у всех транспортов ровно один отсек
	\begin{enumerate}
		\item doAllTransportsHaveExactlyOneBox
		
		У всех транспортов ровно один отсек.
	\end{enumerate}
\end{enumerate}

\subsection{Delivery}

Это подкласс Multiple Depots, и все свойства этого класса задач сохраняются, а также нужны следующие дополнительные свойства:

\begin{enumerate}
	\item Все заказы стартуют из одного места
	\begin{enumerate}
		\item doAllPickupsStartFromSamePlace
	
	     Все события всех заборов расположены в одной и той же локации.
	\end{enumerate}

	\item Нет каких-бы то ни было несовместимостей
	\begin{enumerate}
		\item haveNoPerformerTransportNonCompatibilities
		
		Все исполнители совместимы со всеми транспортами.
		\item haveNoPerformerOrderNonCompatibilities
		
		Все исполнители совместимы со всеми заказами.
		\item haveNoOrderOrderNonCompatibilities	
		
		Все заказы совместимы друг с другом.
		\item haveNoCargoBoxNonCompatibilities
		
		Все грузы совместимы со всеми отсеками.
		\item haveNoTransportLocationNonCompatibilities
		
		Все транспорты совместимы со всеми локациями.
					
	\end{enumerate}
	\item Все транспорты одинаковые
	\begin{enumerate}
		\item doAllTransportsHaveSameShifts
		\item doAllTransportsHaveSameLocations
		
		У всех транспортов одинаковые локации, если взять два разных транспорта, то смен у них поровну и в соответствующих сменах локации старта и финиша у них совпадают (не старт с финишом, а старт одной смены со стартом другой смены).
	\end{enumerate}
	\item Все исполнители одинаковые
	\begin{enumerate}
		\item doAllPerformersHaveSameWorkShifts
	    \item doAllPerformersHaveSameLocations
	    
	    У всех исполнителей одинаковые локации, если взять два разных исполнителя, то смен у них поровну и в соответствующих сменах локации старта и финиша у них совпадают (не старт с финишом, а старт одной смены со стартом другой смены).
	    	    
	\end{enumerate}

\item Все исполнители одинаковые
\begin{enumerate}
	\item doAllPerformersHaveSameWorkShifts
	\item doAllPerformersHaveSameLocations
	
	У всех исполнителей одинаковые локации, если взять два разных исполнителя, то смен у них поровну и в соответствующих сменах локации старта и финиша у них совпадают (не старт с финишом, а старт одной смены со стартом другой смены).
	
\end{enumerate}

\item Нет каких-либо сложных дополнительных ограничений

\begin{enumerate}
	\item haveNoMaxWorkShiftsRestrictions
	
	Ограничения на максимальное число рабочих смен не мешают планированию. У всех исполнителей ограничение на максимальное количество смен не меньше, чем общее количество смен исполнителя.
	\item haveNoMaxTransportsRestrictions
	
	Ограничение на максимальное количество транспортов в локации не мешает планированию. Для каждой локации максимальное теоретически возможное количество транспортов, которые могли бы в ней одновременно находится, не превосходит ограничение на одновременное количество транспортов в локации.
	\item haveNoMaxTotalOrdersRestrictions
	
	Ограничение на максимальное число выполненных заказов не мешает планированию. У всех исполнителей число заказов, которые он может теоретически выполнить за одну смену с учетом совместимостей и продолжительности заказов, не больше, чем ограничение на количество заказов в эту смену.
	\item haveNoMaxTotalLocationsRestrictions
	
	Ограничение на максимальное число посещенных локаций не мешает планированию. У всех исполнителей в каждой смене максимальное дозволенное число посещенных локаций не менбше, чем общее число локаций.
	\item !haveBreaks
	
	Есть ли правила перерывов исполнителей.
\end{enumerate}

\item Можно игнорировать тарификацию и допущения

\begin{enumerate}

\item haveTrivialTariffs

Простая тарификация. Ограничение на продолжительность работы исполнителя не меньше, чем продолжительность смены; все время работы исполнителя тарифицируется одинаково; весь пробег транспорта тарифицируется одинаково.
\item haveTrivialContraventions

Простые допущения. Ровно одна группа; требуется выполнить все заказы, то есть SuccessOrdersQuota 100; SoftWindowsQuota 0.

\end{enumerate}
\end{enumerate}



\section{Описание данных и работа с ними}

\subsection{Исходные данные}

Что же представляет из себя набор данных в представленной задаче.

В целом, это документы (xlsx формата), в которых представлена следующая информация:

\begin{enumerate}
	\item Заказы;
	\item Грузы;
	\item Локации;
	\item Транспортные средства;
	\item Водители;
	\item Передвижения (любые перемещения водителей и транспорта, и вся информация, связанная с этим).	
\end{enumerate}

\subsection{Конвертирование данных в программный формат}

Абсолютно очевидно, что работа с данными формата xlsx весьма затруднительна. Для этого в компании специально была написана утилита, которая называется <<Mission Control Center>>. Эта утилита запускает локальный сервер, где можно делать различные вещи с данными в рамках компании VeeRoute. В том числе и конвертировать из xlsx в удобный json формат. Помимо простого перегона одного в другое, конвертер добавляет некоторые настройки роутинга, маршрутизации и вообще способствует более грамотному представлению данных.
 

\chapterconclusion

В этой главе была рассмотрена предметная область, описаны данные, с которыми происходит работа.
Также поднимается описание бизнес-задачи, которую мы решаем,и ее актуальность для компании.
Рассказывается про данные, с которыми происходит работа.
\documentclass{beamer}
\usetheme{CambridgeUS}

%\documentclass[handout]{beamer}
%\usetheme{Pittsburgh}
%\beamertemplatesolidbackgroundcolor{black!2}
%\setbeamertemplate{footline}[frame number]
%\usepackage{pgfpages}
%%\pgfpagesuselayout{4 on 1}[a4paper,border shrink=5mm,landscape]
%\pgfpagesuselayout{8 on 1}[a4paper,border shrink=5mm]

%%% PACKAGES
\usepackage[russian]{babel}
\usepackage[utf8]{inputenc}
\usepackage{amsmath}
\usepackage{amssymb}
\usepackage{tikz}
\usepackage{graphics}
\usepackage{appendixnumberbeamer}

%%% BEAMER SETTINGS
\setbeamertemplate{navigation symbols}{}
\setbeamertemplate{headline}{}

%%% TIKZ SETTINGS
\usetikzlibrary{fit}

%%% NEW COMMANDS
\def\pitem{\pause \item}

\DeclareGraphicsRule{.1}{mps}{*}{}
\DeclareGraphicsRule{.2}{mps}{*}{}
\DeclareGraphicsRule{.3}{mps}{*}{}
\DeclareGraphicsRule{.4}{mps}{*}{}
\DeclareGraphicsRule{.5}{mps}{*}{}
\DeclareGraphicsRule{.6}{mps}{*}{}

%\includeonlyframes{current} % leaves only the given frames

\title[Метрика качества]{Определение метрики качества для известного типа задач}
%\transduration{20}
\author[Шведов Денис]{Шведов Денис}
\institute[]{Национальный исследовательский университет информационных технологий, механики и оптики}
\date{}

\begin{document}

\begin{frame}
%\transduration{20}
\begin{center}
{\scriptsize Санкт-Петербургский национальный исследовательский университет \\ информационных технологий, механики и оптики}

\vspace{1cm}

{\scriptsize Факультет информационных технологий и программирования

Кафедра компьютерных технологий}

\vspace{1cm}

\vbox{\large\bfseries

Определение метрики качества для известного типа задач}

\vspace{1cm}

{\large Шведов Денис Владимирович \\}
{\large Группа M3436}


\vspace{1cm}

{\large Научный руководитель: к.ф.-м.н. доцент кафедры КТ \\}
{\large А.~А.~Фильченков}


\end{center}
\end{frame}

%%%%%%%%%%%%%%%%%%%%%%%%%%%%%%%%%%%%%%%%%%%%%%%%%%%%%%%%%%%%%%%%%%%%%%%%%%%%%%%%%%%%%%%%%%%%%%%%%%%%%%%%%%%%%%%%%%%%%%%%
%\frame[label=title]{\titlepage}


\begin{frame}{Решаемая проблема}
%\transduration{20}
\begin{block}{Цель исследования}
Компания VeeRoute занимается разработкой специальных алгоритмов, которые позволяют выстраивать маршруты в реальном времени.	
По данным, предоставленным компанией, построить классификатор, который определит тип новых наборов данных. 
\begin{itemize}
\item Грамотно подобрать признаки для классификатора
\item Выбрать лучший алгоритм классификации для решения поставленной задачи
\item Протестировать на разных наборах данных и добиться хорошего результата
\end{itemize}
\end{block}
\end{frame}

\begin{frame}{Решаемая проблема}
%\transduration{20}
\begin{block}{Актуальность}
\begin{itemize}
\item В компанию приходит заказчик с определенной бизнес-задачей и своими датасетами.
\item При этом в компании не знают, корректную ли вообще задачу ставит заказчик.
\item Вполне возможно, что он ошибается.
\end{itemize}
\end{block}
\end{frame}

\begin{frame}{Решаемая проблема}
%\transduration{20}
\begin{block}{Актуальность}
	\begin{itemize}
		\item Проверяется похожесть нового датасета, к одной из групп, ранее отобранных.
		\item Можно посмотреть, какая бизнес-задача решалась на уже известных данных и сравнить с тем, что предоставил заказчик.
		\item Таким образом, можно еще на первом шаге устранить ошибку и переформулировать задачу.
	\end{itemize}
\end{block}
\end{frame}

\begin{frame}{Решение}
%\transduration{20}
\begin{block}{Описание исходных данных}
Каждый набор данных представляет собой информацию о
\begin{itemize}
\item Заказах
\item Грузах
\item Водителях и их передвижениях
\item Траспортных средствах и их передвижениях
\item Локациях
\end{itemize}
\end{block}
\end{frame}

\begin{frame}{Решение}
%\transduration{20}
\begin{block}{Выбор признаков}
Было выбрано примерно более 50 признаков для построения классификатора.

Их можно разбить на следующие категории.

\begin{itemize}
\item Количественные признаки (количество заказов, транспортных средств, водителей, локаций в одном наборе данных)
\item Матрицы совместимости (например, между исполнителем-транспортом, транспортом-локацией, исполнителем-заказом, грузом-отсеком траспнортного средства и т.д.)
\item Статистические признаки (например, среднее количество рабочих смен для исполнителей, среднее количество грузов в заказах, средняя длительность временного отрезка в минутах (для исполнителя) \ длительность пути в метрах для транспорта и т.д.)
\item Геокоординаты
\end{itemize}
\end{block}
\end{frame}

\begin{frame}{Решение}
%\transduration{20}
\begin{block}{Типы многоклассовой классификации}
При решении использовались следующие типы классификаций

\begin{itemize}
\item Дерево решений (C4.5)
\item Многоклассовый метод опорных векторов
\item Многоклассовая логиститическая регрессия
\item Random Forest
\end{itemize}
\end{block}
\end{frame}

\begin{frame}{Решение}
%\transduration{20}
\begin{block}{Описание решения }
\begin{itemize}
	\item Для исследования брались 4 разных класса наборов данных, не связанных друг с другом.В каждом классе выбиралось одинаковое числов наборов для рассмотрения
	\item В качестве обучающей выборки ~--- 50 \% от каждого класса наборов данных
	\item В качестве тестовой выборки ~--- 50 \% оставшихся данных
	\item При каждом запуске данные перемешивались.
	\item Далее будут рассмотрены результаты нескольких категорий тестирования. В каждой категории какое-то количество признаков зашумлено и не участвует в построении классификатора.
\end{itemize}
\end{block}
\end{frame}


\begin{frame}{Результаты}
%\transduration{20}
\begin{block}{Все признаки используются}
\begin{table}[!h]
	\centering
	\caption{Средняя $F_1$ мера}
	\label{my-label}
	\begin{tabular}{|*{18}{c|}}\hline
		Дерево Решений & 0.949\\\hline
		SVM &  0.949  \\\hline
		LogReg & 0.949  \\\hline
		RandomForest &0.949 \\\hline
	\end{tabular}
\end{table}
\end{block}
\end{frame}

\begin{frame}{Результаты}
%\transduration{20}
\begin{block}{Не используется признак количества заказов}
	\begin{table}[!h]
		\centering
		\caption{Средняя $F_1$ мера}
		\label{my-label}
		\begin{tabular}{|*{18}{c|}}\hline
			Дерево Решений & 1.0\\\hline
			SVM &  0.949  \\\hline
			LogReg & 0.778  \\\hline
			RandomForest &0.845 \\\hline
		\end{tabular}
	\end{table}
\end{block}
\end{frame}

\begin{frame}{Результаты}
%\transduration{20}
\begin{block}{Не используются признаки количества заказов,транспорта и водителей}
	\begin{table}[!h]
		\centering
		\caption{Средняя $F_1$ мера}
		\label{my-label}
		\begin{tabular}{|*{18}{c|}}\hline
			Дерево Решений & 0.949\\\hline
			SVM &  0.949  \\\hline
			LogReg & 0.896  \\\hline
			RandomForest &0.899 \\\hline
		\end{tabular}
	\end{table}
\end{block}
\end{frame}

\begin{frame}{Результаты}
%\transduration{20}
\begin{block}{Не используется половина рандомных признаков}
	\begin{table}[!h]
		\centering
		\caption{Средняя $F_1$ мера}
		\label{my-label}
		\begin{tabular}{|*{18}{c|}}\hline
			Дерево Решений & 0.949\\\hline
			SVM &  0.949  \\\hline
			LogReg & 0.949  \\\hline
			RandomForest &0.899 \\\hline
		\end{tabular}
	\end{table}
\end{block}
\end{frame}

\begin{frame}{Результат}
%\transduration{20}
\begin{block}{Выводы}
\begin{itemize}
\item Наиболее лучший результат по итогам тестирования был показан при использовании метода "Дерево Решений".$F_1$ мера составила примерно 0.95
\item В целом, выбранные признаки хорошо классифицируют наборы данных
%\item {Выявлена зависимость от числа точек, размерности и фронтов}
%\item {Есть результат на реальных данных}
\end{itemize}
\end{block}
\end{frame}

\begin{frame}{}
\begin{center}
Спасибо за внимание!
\end{center}
\end{frame}

\appendix

%\begin{frame}{Дополнительные материалы}
%
%\end{frame}

\end{document}